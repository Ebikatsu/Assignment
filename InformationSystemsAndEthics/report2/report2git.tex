\documentclass[a4paper]{jsarticle}
\title{情報システムと職業倫理レポート課題}
\date{2016/11/23}
\begin{document}
\maketitle
\section{自分の特徴}
\begin{itemize}
	\item プログラミングが好き。特にC言語、C++、アセンブリが好き。
	%\item 古いものが好き。(聖書、コーラン、古事記、日本書紀をよく読む、ノートは旧字体で書くなど)
	%\item 古い価値観を重んじる。(神道、儒教など)
	\item 流行に乗らない。
	\item 哲学的なことを考えるのが好き。
	\item 深く考えすぎてしまう。決断力がない。
	\item CUIが好き。GUIが嫌い。
	\item 自分自身の考えをしっかり持っている。意見を言うときに根拠がちゃんとある。
	\item 好きなことがたくさんある。
	\item 語学が好き。(英語を除く)
	\item 英語が嫌い。
	\item 物理が得意。
	\item 集合論が好き。
	\item 考えることが好き。
	%\item チベット音楽が好き。
	\item Linuxが好き。
\end{itemize}
\section{ミラクルリナックスについて}
\subsection{ミラクルリナックスを選んだ理由}
一番の魅力はミラクルリナックスの研究・開発内容です。Linuxディストリビューションの開発や、組み込み機器用にカーネルレベルからカスタマイズされた組み込みLinuxの開発など、自分もやってみたいと思うような様々な開発が行われています。
\subsection{主なサービス}
\begin{itemize}
	\item LinuxのOSを日本の企業ユーザ向けに開発・販売
	\item 企業向け統合監視ツール
	\item 企業システムの統合監視ツール
	\item カーネル技術を生かした組み込み事業
\end{itemize}
公共・通信のようなインフラシステム、カーナビゲーション、医療機器、デジタルサイネージ、映像配信機器などの特殊用途の専用機器などに利用されています。
\subsection{どのような技能が求められるか、自分の特徴とどのようにマッチングしているか}
必須経験・スキルとして、C言語、Java、Python等を利用したソフトウェア開発経験、任意経験・スキルとして、英語文章の作成・読解、オープンソース・ソフトウェアコミュニティでの活動などが挙げられています。私はプログラミングが好きで授業以外でも様々なプログラムを作成したことがあります。また組み込み系の開発ではアセンブリが使われることもあるので、アセンブリの知識を生かすこともできるでしょう。英語は苦手ですが勉強すれば何とかなります。
\begin{thebibliography}{1}
	\item ミラクルリナックス
		\textless
		https://www.miraclelinux.com
		\textgreater
\end{thebibliography}
\end{document}
