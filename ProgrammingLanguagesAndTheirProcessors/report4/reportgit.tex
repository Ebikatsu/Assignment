\documentclass[a4paper,12pt]{jarticle}
\usepackage{comment}
\usepackage[top=20truemm,bottom=20truemm,left=20truemm,right=20truemm]{geometry}
\usepackage{here}
\usepackage[utf8]{inputenc}
\usepackage{listings,jlisting}
\lstset{tabsize = 8}
\usepackage{slashbox}
\newcommand{\dotgt}{\ooalign{$>$\crcr\hss$\cdot$\hss}}
\newcommand{\dotlt}{\ooalign{$<$\crcr\hss$\cdot$\hss}}
\begin{document}
\section*{課題1}
\begin{comment}
教科書表4.2の演算子順位行列が表す算術式に対して、次の2点の追加を考える。
\begin{enumerate}
 \renewcommand{\labelenumi}{(\arabic{enumi})}
 \item 構造体修飾子.のついた参照が許されるようにする(例えばx.aのような参照)。
 \item 右辺の式の結果を左辺へ代入する演算子$\leftarrow$を追加する。優先順位は例えばx.a*y.bは((x.a)*(y.b))を意味するものとする。
\end{enumerate}
これらを施したものについて、演算順位行列を示せ。
\begin{table}[H]
 \begin{center}
  \caption{教科書表4.2}
  \begin{tabular}{|c|cccccccc|}\hline
	\backslashbox{左}{右}&	$+$&	$-$&	$*$&	$/$&	$($&	$)$&		識別子&	終\\		\hline
	始&			\dotlt&	\dotlt&	\dotlt&	\dotlt&	\dotlt&	&		\dotlt&	$\doteq$\\
	$+$&			\dotgt&	\dotgt&	\dotlt&	\dotlt&	\dotlt&	\dotgt&		\dotlt&	\dotgt\\	
	$-$&			\dotgt&	\dotgt&	\dotlt&	\dotlt&	\dotlt&	\dotgt&		\dotlt&	\dotgt\\	
	$*$&			\dotgt&	\dotgt&	\dotgt&	\dotgt&	\dotlt&	\dotgt&		\dotlt&	\dotgt\\	
	$/$&			\dotgt&	\dotgt&	\dotgt&	\dotgt&	\dotlt&	\dotgt&		\dotlt&	\dotgt\\	
	$($&			\dotlt&	\dotlt&	\dotlt&	\dotlt&	\dotlt&	$\doteq$&	\dotlt&	\\	
	識別子&			\dotgt&	\dotgt&	\dotgt&	\dotgt&	&	\dotgt&		&	\dotgt\\	\hline
  \end{tabular}
 \end{center}
\end{table}
\end{comment}
\begin{table}[H]
 \begin{center}
  \begin{tabular}{|c|cccccccccc|}\hline
	\backslashbox{左}{右}&	$+$&	$-$&	$*$&	$/$&	$($&	$)$&		識別子&	.&	$\leftarrow$&	終\\		\hline
	始&			\dotlt&	\dotlt&	\dotlt&	\dotlt&	\dotlt&	&		\dotlt&	\dotlt&	\dotlt&	$\doteq$\\
	$+$&			\dotgt&	\dotgt&	\dotlt&	\dotlt&	\dotlt&	\dotgt&		\dotlt&	\dotlt&	&	\dotgt\\	
	$-$&			\dotgt&	\dotgt&	\dotlt&	\dotlt&	\dotlt&	\dotgt&		\dotlt&	\dotlt&	&	\dotgt\\	
	$*$&			\dotgt&	\dotgt&	\dotgt&	\dotgt&	\dotlt&	\dotgt&		\dotlt&	\dotlt&	&	\dotgt\\	
	$/$&			\dotgt&	\dotgt&	\dotgt&	\dotgt&	\dotlt&	\dotgt&		\dotlt&	\dotlt&	&	\dotgt\\	
	$($&			\dotlt&	\dotlt&	\dotlt&	\dotlt&	\dotlt&	$\doteq$&	\dotlt&	\dotlt&	&	\\	
	識別子&			\dotgt&	\dotgt&	\dotgt&	\dotgt&	&	\dotgt&		&	\dotgt&	\dotgt&	\dotgt\\
	.&			\dotgt&	\dotgt&	\dotgt&	\dotgt&	&	\dotgt&		\dotlt&	\dotgt&	\dotgt&	\dotgt\\
	$\leftarrow$&		\dotlt&	\dotlt&	\dotlt&	\dotlt&	&\dotlt	&		\dotlt&	\dotlt&	&	\dotgt\\	\hline
  \end{tabular}
 \end{center}
\end{table}
\section*{課題2}
\begin{comment}
課題1で代入文\\
s = s + (a.x - b.x) * (c.y + d.y)\\
を演算子順位法で構文解析する過程を、教科書の図4.1のようにスタックの動きをトレースする形で示せ。
\end{comment}
\fontsize{10pt}{0pt}\selectfont
\begin{lstlisting}[caption=構文解析の過程]
スタック				入力の残り
始					s = s + (a.x - b.x) * (c.y + d.y)終
始 s					= s + (a.x - b.x) * (c.y + d.y)終
始 r1 					= s + (a.x - b.x) * (c.y + d.y)終
始 r1 =					s + (a.x - b.x) * (c.y + d.y)終
始 r1 = s				+ (a.x - b.x) * (c.y + d.y)終
始 r1 = r1				+ (a.x - b.x) * (c.y + d.y)終
始 r1 = r1 +				(a.x - b.x) * (c.y + d.y)終
始 r1 = r1 + (				a.x - b.x) * (c.y + d.y)終
始 r1 = r1 + ( a			.x - b.x) * (c.y + d.y)終
始 r1 = r1 + ( r2			.x - b.x) * (c.y + d.y)終
始 r1 = r1 + ( r2 .			x - b.x) * (c.y + d.y)終
始 r1 = r1 + ( r2 . x			- b.x) * (c.y + d.y)終
始 r1 = r1 + ( r2 . r3			- b.x) * (c.y + d.y)終
始 r1 = r1 + ( r4			- b.x) * (c.y + d.y)終
始 r1 = r1 + ( r4 -			b.x) * (c.y + d.y)終
始 r1 = r1 + ( r4 - b			.x) * (c.y + d.y)終
始 r1 = r1 + ( r4 - r5			.x) * (c.y + d.y)終
始 r1 = r1 + ( r4 - r5 .		x) * (c.y + d.y)終
始 r1 = r1 + ( r4 - r5 . x		) * (c.y + d.y)終
始 r1 = r1 + ( r4 - r5 . r3		) * (c.y + d.y)終
始 r1 = r1 + ( r4 - r6			) * (c.y + d.y)終
始 r1 = r1 + ( r7			) * (c.y + d.y)終
始 r1 = r1 + r7				* (c.y + d.y)終
始 r1 = r1 + r7 *			(c.y + d.y)終
始 r1 = r1 + r7 * (			c.y + d.y)終
始 r1 = r1 + r7 * ( c			.y + d.y)終
始 r1 = r1 + r7 * ( r8			.y + d.y)終
始 r1 = r1 + r7 * ( r8 .		y + d.y)終
始 r1 = r1 + r7 * ( r8 . y		+ d.y)終
始 r1 = r1 + r7 * ( r8 . r9		+ d.y)終
始 r1 = r1 + r7 * ( r10			+ d.y)終
始 r1 = r1 + r7 * ( r10 +		d.y)終
始 r1 = r1 + r7 * ( r10 + d		.y)終
始 r1 = r1 + r7 * ( r10 + r11		.y)終
始 r1 = r1 + r7 * ( r10 + r11 .		y)終
始 r1 = r1 + r7 * ( r10 + r11 . y	)終
始 r1 = r1 + r7 * ( r10 + r11 . r9	)終
始 r1 = r1 + r7 * ( r10 + r12		)終
始 r1 = r1 + r7 * ( r13			)終
始 r1 = r1 + r7 * r13			終
始 r1 = r1 + r14			終
始 r1 = r15				終
始					終

\end{lstlisting}
\fontsize{12pt}{0pt}\selectfont
\section*{課題3}
\begin{comment}
課題2の代入文をポーランド記法(逆ポーランド記法)で表現せよ。
\end{comment}
s s a x . b x . - c y . d y . + * + =
\end{document}
