\documentclass[a4paper]{jarticle}
\usepackage{comment}
\usepackage[top=20truemm,bottom=20truemm,left=20truemm,right=20truemm]{geometry}
\usepackage[utf8]{inputenc}
\title{コンピュータサイエンス講究}
\author{学生番号 氏名}
\date{\today}
\begin{document}
\maketitle
\section{自分の夢について}
\begin{comment}
私の夢は修行僧になる事です。
20年間生きてき
いろんなことをやりました。水泳、ボーイスカウト、まなたび、
辛いこともたくさんあった。
世界に対していろんな疑問を持つ。ほかの命を奪わなければ生きていけないという自然の摂理、世界が存在する必要性を感じない、「無」のほうが何の問題も起きないし、簡単だし、楽。
こういう疑問を解決するためには、「生きる」という道しかない。だから生きる。
「苦」というものに向き合い、生涯「苦」とたたかい続けてきた釈尊の生き方にあこがれる。
多くの人は、どうしたら幸せになれるかということばかり考えているが、仏教ではなぜ人はこんなに苦しまなければいけないのかという視点がある。
自分だけ幸せになって、この世の苦しみから目をそらすような人にはなりたくない。
仏教の中でも、「禅」に興味がある。禅はとても実践的なもので、座禅や読経などの特別な行為のみを修行ととらえるのではなく、食事や掃除などを含めた生活や人生そのものが修行であるとする考え方
一つのことに集中することで心に雑念の生じるすきをなくす。
修行僧を目指して、小さなことから「禅」を実践していきたい。
父に聖書を買ってもらった。キリスト教徒ではないが。それ以来いろんな宗教について調べるのが趣味になった。
\end{comment}
私の夢は修行僧になる事です。

私はこの世に生まれてきて20年になります。今までいろんなことがあって、楽しいこともたくさんあり辛いこともたくさんありました。私は昔からいろんなことに興味を持つ人で、車が動く仕組みが気になってエンジンに関する本を読みあさった時期もあれば、「聖書がほしい」といって父に聖書を買ってもらって毎日夢中になって読んでいた時期もありました。
最近特にそうですが、哲学的なことを考えることが好きです。「自分とはだれなのか」とか「他人の心は存在するのか」とか、考えてもきりがないのはわかっていてもつい考えてしまします。
最近は特に世界が存在することに関して疑問に思います。世界が存在する必要性が分からないし、「無」の方が何の問題も起きず、単純明快で、楽です。なのに世界は、存在するという、とてつもなく面倒なことをしているのです。

こういった世界に対する疑問を解決するためには、この世界の中で、懸命に生きて、答えを探すしかないのだろうと思います。私は聖書を買ってもらってから、世界中にあるいろんな宗教について調べるのが趣味でした。その時はただ、「こんな宗教もあるのか」ということしか思っていませんでしたが、大学生になって一人暮らしを始めて、すべて自分でやっていかなければならない状況に置かれたときに、宗教の意味をより深く理解できるようになりました。例えば禅宗では修行というのは座禅や読経などの特別な行為を指すのではなく、食事や掃除などを含めた日常生活や人生そのものであり、一つのことに集中して取り組むことで心に雑念の生じる隙をなくすという考え方があり、家事や勉強をしている時にそれを実践してその感覚をすこし実感してみたりしました。
多くの宗教の中で仏教は自分に合っていると思います。仏教は「どうすれば幸せになれるのか」ではなく、「なぜ人はこんなに苦しまなければならないのか」という視点があります。私にとってもそれが大きな悩みなのです。

大学を卒業したら大学院に進学して、学んだことを生かしたいので一旦はIT企業に就職すると思います。どのタイミングで修行僧になるかは決めていませんが、IT企業で働く時でも禅の精神を忘れないことが大切です。
これから修行僧を目指して、小さなことから「禅」を実践していきたいです。
\end{document}
