\documentclass[a4paper]{jarticle}
\title{まとめ}
\author{名前}
\usepackage[top=20truemm,bottom=20truemm,left=20truemm,right=20truemm]{geometry}
\begin{document}
\maketitle
\section{}
\section{}
\section{}
\section{}
\section{形式文法}
\subsection{形式文法の形}
\begin{equation}
G = < N , \Sigma , P , S >
\end{equation}
但し
\begin{equation}
N \cap \Sigma = \phi
\end{equation}
\begin{equation}
P \supseteq \bigl\{ A \to B | A , B \in \left( N \cup \Sigma \right) ^* \bigr\}
\end{equation}
\begin{equation}
S \in N
\end{equation}
$G$:形式文法\\
$N$:非終端記号の有限集合\\
$\sigma$:終端器号の有限集合\\
$P$:生成規則の有限集合\\
$S$:初期非終端記号
\subsubsection{適用可能}
形式文法$ G = < N , \Sigma , P , S > $において、記号列$ \xi _1 , \alpha , \xi _2 \in \left( N \cup \Sigma \right) ^+ $からなる記号列$ \xi _1 \alpha \xi _2 $が適応可能であることを
\begin{equation}
\xi _1 \alpha \xi _2 \Rightarrow _G \xi _1 \beta \xi _2
\end{equation}
と書き、
\begin{equation}
\xi _1 \alpha \xi _2 \Rightarrow _G \xi _1 \beta \xi _2 \iff \alpha \to \beta \in P
\end{equation}
と定義する。
\end{document}
