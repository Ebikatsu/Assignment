\documentclass[a4paper]{jarticle}
\title{まとめ}
\author{名前}
\usepackage{amsmath}
\usepackage{amsfonts}
\usepackage{amsthm}
\usepackage{ascmac}
\usepackage{comment}
\usepackage[top=20truemm,bottom=20truemm,left=20truemm,right=20truemm]{geometry}
\begin{document}
\maketitle
\section{}
\section{}
\section{}
\section{}
\section{形式文法}
\subsection{形式文法の形}
\begin{equation}
G = < N , \Sigma , P , S >
\end{equation}
但し
\begin{equation}
N \cap \Sigma = \phi
\end{equation}
\begin{equation}
P \subseteq \bigl\{ A \to B | A , B \in \left( N \cup \Sigma \right) ^* \bigr\}
\end{equation}
\begin{equation}
S \in N
\end{equation}
$G$:形式文法\\
$N$:非終端記号の有限集合\\
$\sigma$:終端記号の有限集合\\
$P$:生成規則の有限集合\\
$S$:初期非終端記号
\subsection{適用可能}
形式文法$ G = < N , \Sigma , P , S > $において、記号列$ \xi _1 , \alpha , \xi _2 \in \left( N \cup \Sigma \right) ^+ $からなる記号列$ \xi _1 \alpha \xi _2 $に生成規則$ \alpha \to \beta $が適応可能であることを
\begin{equation}
\xi _1 \alpha \xi _2 \Rightarrow _G \xi _1 \beta \xi _2
\end{equation}
と書き、
\begin{equation}
\xi _1 \alpha \xi _2 \Rightarrow _G \xi _1 \beta \xi _2 \iff \alpha \to \beta \in P
\end{equation}
と定義する。\\
但し、記号の有限集合$ \Sigma $に対して、$ \Sigma ^+ $は$ \Sigma $の要素を一文字以上並べた語の集合を意味する。
\subsection{導出}
記号列$ \xi $に、形式文法$ G $の生成規則を有限回適用させるとこで記号列$ \zeta $が得られることを
\begin{equation}
\xi \Rightarrow _G ^* \zeta
\end{equation}
と書く。
\subsection{言語}
\begin{equation}
L \left( G \right) = \bigl\{ \omega \in \Sigma ^* | S \Rightarrow _G ^* \omega \bigr\}
\end{equation}
を$ G = < N , \Sigma , P , S > $の言語という。\\
但し、記号の有限集合$ \Sigma $に対して、$ \Sigma ^* $は$ \Sigma $の要素を0文字以上並べた語の集合を意味する。
\subsection{等価}
2つの形式文法$ G , H $が等しいことを
\begin{equation}
G = H
\end{equation}
と書き、
\begin{equation}
G = H \iff L \left( G \right) = L \left( H \right)
\end{equation}
と定義する。
\subsection{言語クラス}
形式文法の集合$ \mathbb{G} $について、
\begin{equation}
\mathcal{L} \left( \mathbb{G} \right) = \bigl\{ L \left( G \right) | G \in \mathbb{G} \bigr\}
\end{equation}
を$ \mathbb{G} $の言語クラスという。
\subsection{形式文法の型}
4つの形式文法の型、句構造文法$psg$、文脈依存文法$csg$、文脈自由文法$cfg$、正規文法$rg$を次のように定義する。但し、$ \mathbb{G} $はすべての形式文法の集合である。
\begin{eqnarray}
psg & = & \mathbb{G} \\
csg & = & \bigl\{ < N , \Sigma , P , S > \in \mathbb{G} | P \subseteq \bigl\{ \alpha _0 A \alpha _1 \to \alpha _0 \gamma \alpha _1 | \alpha _0 , \alpha _1 \in \left( N \cup \Sigma \right) ^* \land A \in N \land \gamma \in \left( N \cup \Sigma \right) ^+ \bigr\} \bigr\} \\
cfg & = & \bigl\{ < N , \Sigma , P , S > \in \mathbb{G} | P \subseteq \bigl\{ A \to \beta | A \in N \land \beta \in \left( N \cup \Sigma \right) ^+ \bigr\} \bigr\} \\
rg & = & \bigl\{ < N , \Sigma , P , S > \in \mathbb{G} | P \subseteq \bigl\{ A \to a | A \in N \land a \in \Sigma \bigr\} \cup \bigl\{ A \to aB | A , B \in N \land a \in \Sigma \land \bigr\} \bigr\}
\end{eqnarray}
\subsection{空語$ \epsilon $の生成}
上の定義では、$csg$、$cfg$、$rg$の形式文法では、文字をひとつも含まない空語$ \epsilon $を生成することが出来ない。\\
$ \epsilon $を生成できない形式文法$G$に対して、
\begin{equation}
G' = < N \cup \bigl\{ S_0 \bigr\} , \Sigma , P \cup \bigl\{ S_0 \to \epsilon \bigr\} \cup \bigl\{ S_0 \to \beta | S \to \beta \in P \bigr\} , S_0 >
\end{equation}
は
\begin{equation}
L \left( G' \right) = L \left( G \right) \cup \bigl\{ \epsilon \bigr\}
\end{equation}
を満たす。\\
このようにして、$ \epsilon $を生成できない形式文法の言語に$ \epsilon $を追加することができる。
\subsection{チョムスキーの標準形}
\subsubsection{チョムスキーの標準形とは}
形式文法$ G = < N , \Sigma , P , S > $が
\begin{equation}
P \subseteq \bigl\{ A \to a | A \in N \land a \in \Sigma \bigr\} \cup \bigl\{ A \to BC | A , B , C \in N \bigr\}
\end{equation}
を満たしているとき、$ G $はチョムスキーの標準形であるという。\\
$cfg$の形式文法は全てチョムスキー標準形で表すことができる。\\
\subsubsection{任意の$cfg$の形式文法をチョムスキーの標準形にするアルゴリズム}
$ \forall G = < N , \Sigma , P , S > \in cfg $から、チョムスキーの標準形である$ G' = < N' , \Sigma , P' , S > $を構成する。について、先ず生成規則の右辺に現れることのない終端文字、非終端文字と、その非終端文字を左辺に持つ生成規則を削除する。つまり、無駄な文字や生成規則を削除する。次に、$ A , B \in N $である$ A \to B $が$ P $の要素にある場合は、$ B $を左辺に持つ生成規則それぞれと結合させてしまう。次に、$ N' $に$N$の要素をすべて追加しておく。次に$ P \cap \left( \bigl\{ A \to a | A \in N \land a \in \Sigma \bigr\} \cup \bigl\{ A \to BC | A , B , C \in N \bigr\} \right) $の要素である生成規則はそのまま$ P' $の要素とする。$ P - \left( \bigl\{ A \to a | A \in N \land a \in \Sigma \bigr\} \cup \bigl\{ A \to BC | A , B , C \in N \bigr\} \right) $の要素である生成規則については、先ずその右辺に含まれる終端文字$ n_i $を対応する新しい非終端文字$ B_i \in N' $に置き換え、$B_i \to n_i $を$ P' $の要素として加える。その置き換えを終えた生成規則$ A \to B_1 B_2 \cdots B_k $について、新しい非終端文字$ X_1 , X_2 , \cdots , X_{ k - 2 } \in N' $を用意して、新しい生成規則$ A \to B_1 X_1 , X_1 \to B_2 X_2 , \cdots , X_{ k - 2 } \to B_{ k - 1 } B_k \in P' $を追加する。$ P $の全要素について以上の操作を行って$ P' $に生成規則を追加すれば$ G' $を完成させることができる。
\subsection{グライバッハの標準形}
\subsubsection{グライバッハの標準形とは}
形式文法$ G = < N , \Sigma , P , S > $が
\begin{equation}
P \subseteq \bigl\{ A \to a \alpha | A \in N \land a \in \Sigma \land \alpha \in N ^* \bigr\}
\end{equation}
を満たしているとき、$ G $はグライバッハの標準形であるという。\\
\subsubsection{任意のチョムスキー標準形の形式文法をグライバッハの標準形にするアルゴリズム}
チョムスキーの標準形である$ \forall G = < N , \Sigma , P , S > \in cfg $から、グライバッハの標準形である$ G' = < N' , \Sigma , P' , S > $を構成する。\\
このアルゴリズムでは、生成規則に関して以下の二種類の置換えを行う。
\begin{itembox}[l]{置き換え処理1}
$ A \to B \alpha \in P $であるとする。このとき$ G $はチョムスキーの標準形であるから$ A , B , \alpha \in N $である。また、$ B $を左辺とする$ P $の要素が$ r $個あり、$ B \to \beta _i \left( 1 \leq i \leq r \right) $とする。これについても、$ G $はチョムスキーの標準形であるから$ \beta _i \in N ^2 \cup \Sigma $である。ここで$ N^n $は$ N $の要素を$ n $文字並べた記号列の集合である。このとき、$ A \to B \alpha $と$ B \to \beta _i $を$ A \to \beta _i \alpha $に置き換える。つまり$ A \to B \alpha $を削除して$ A \to \beta _i \alpha $を追加する。但し、他の生成規則の右辺に$ B $が現れている可能性があるので、$ B \to \beta _i $は削除してはならない。
\end{itembox}
\begin{proof}
チョムスキーの標準形である$G$と、その生成規則$ A \to B \alpha $に対して上の置き換え処理を適用した$G'$が等価であることの証明\\
$ G $と$ G' $がそれぞれ語$ \omega $と$ \omega ' $に対して生成規則を適用させる。2つの語は初期状態で$ \omega = \omega ' $が成り立つとする。\\
$ G $が$ \omega $の$ A $でない部分に対してある生成規則を適用させた場合、$ G $から$ G' $が構成される際に$ A \to B \alpha $以外の生成規則は削除されていないため、$ G' $も同じ生成規則を$ \omega ' $に対して適用することができる。$ G $が$ \omega $内の$ A $に対して$ A \to B \alpha $を適用した時、$ B \in N $より、$ G $は語を完成するまでに必ず$ B $を左辺に取る生成規則を適用しなければならない。$ G $がその$ B $に対して$ B \to \beta _i $を適用した時、$ G' $は$ \omega ' $に対して$ A \to \beta _i $を適用し、更に$ G $が$ \omega $に$ A \to B \alpha $を適用させてから$ B \to \beta _i $を適用させるまでの間に行われた適用を$ G' $が$ \omega ' $に対して行えば、再び$ \omega ' = \omega $となる。よって$ L \left( G \right) = L \left( G ' \right) $である。よって$ G = G ' $である。
\end{proof}
\begin{itembox}[l]{置き換え処理2}
左辺が$ A $で右辺の左端も$ A $である生成規則が$ r $個あるとき、それらの生成規則を$ A \to A \alpha _i \left( 1 \leq i \leq r , A \in N , \alpha _i \in N \right)$とする。また、左辺が$ A $で右辺の左端が$ A $でない生成規則が$ s $個あるとき、それらの生成規則を$ A \to \beta _j \left( 1 \leq j \leq s , A \in N , \beta _j \in N \cup \Sigma \right)$とする。このとき、新しい非終端記号$ Z $を導入して$ A \to A \alpha _i $を$ A \to \beta _j Z , Z \to \alpha _i , Z \to \alpha Z$に置き換える。
\end{itembox}
\begin{proof}
チョムスキーの標準形である$G$と、上の置き換え処理を適用した$G'$が等価であることの証明\\
$ G $と$ G' $がそれぞれ語$ \omega $と$ \omega ' $に対して生成規則を適用させる。2つの語は初期状態で$ \omega = \omega ' $が成り立つとする。\\
	$ G $が$ \omega $内の$ A $でない部分に対してある生成規則を適用させた場合、$ G ' $にも同じ生成規則が含まれているためそれを適用することによってふたたび$ \omega = \omega ' $を成り立たせることができる。$ G $が$ \omega $の$ A $に対して$ A \to \beta _j $を適用した場合、$ G ' $にも同じ生成規則が含まれているためそれを適用することによってふたたび$ \omega = \omega ' $を成り立たせることができる。$ G $が$ \omega $の$ A $に対して$ A \to A \alpha _i $を適用した場合、$ A $は非終端記号であるから左辺の$ A $に対してさらに$ A \to A \alpha _i $( $ i $ の値は異なっていてもよい )を$ 0 $回以上適用し、最後に$ A \to \beta _j $を適用することとなる。このとき$ G ' $がまず$ A \to \beta _j Z $を適用し、次に$ Z \to \alpha _i Z $($ i $の値は異なっていてもよい。$ G $が適用した生成規則に合わせて、$ \omega = \omega ' $となるようにする。)を$ 0 $回以上適用し、最後に$ Z \to \alpha _i $を適用することによって再び$ \omega = \omega ' $を成り立たせることができる。よって$ L \left( G \right) = L \left( G ' \right) $である。よって$ G = G ' $である。
\end{proof}
まず、$ G ' $に$ G $を代入し、チョムスキーの標準形である$ G ' $の非終端記号に番号を付け、$ N ' = \bigl\{ A_i | i \in \mathbb{N} \land 1 \leq i \leq m \bigr\} $とする。置き換え処理1を使って、$ A _i \to A _j \alpha \in P ' \Rightarrow i < j \left( \alpha \in N ^+ \right) $が満たされるように変換する。
\begin{itembox}[l]{変換作業1}
まず、$ A _1 $について、$ A _1 \to A _j \alpha \left( \alpha \in N \subseteq N ^+ \right) $のとき、$ 1 = j $ならば新しい非終端文字$ Z _1 $を導入して置き換え処理2を適用することによって解決できる。$ 1 < j $ならば変換する必要はない。次に、$ \forall i \in \mathbb{N} , 1 \leq i \leq k \land A _i \to A _j \alpha \in P ' \Rightarrow i < j $が満たされるとき、$ A _{k + 1} \to A _j \alpha \land k + 1 > j $を満たす生成規則は変換しなければならない。この生成規則に高々$ k + 1 - j $回置き換え処理1を適用すると生成規則$ A _{ k + 1 } \to A _l \gamma \left( \gamma \in N ^+ , k + 1 \leq l \right) $を得ることができる。$ k + 1 = l $のときは、新しい非終端文字$ Z _{ k + 1 } $を導入して置き換え処理2を適用することによって解決できる。
\end{itembox}
$ k $に$ 1 $から$ m $までの自然数を代入して変換作業1を行うことによって、
\begin{equation}
	P' \subseteq \bigl\{ A _i \to a \alpha | a \in \Sigma \land \alpha \in N' \bigr\} \cup \bigl\{ A _i \to A _j \alpha | i < j \land \alpha \in N' \bigr\} \cup \bigl\{ Z _i \to \alpha | \alpha \in N' \bigr\}
\end{equation}
となる。
\subsection{黒田の標準形}
\subsection{ブックの標準形}
\end{document}
