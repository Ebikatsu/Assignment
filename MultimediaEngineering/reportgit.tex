\documentclass[a4paper]{jsarticle}
\title{マルチメディア工学}
\author{学生番号 氏名 \\ 学生番号 氏名}
\date{2016/12/24}
\usepackage[top=20truemm,bottom=20truemm,left=20truemm,right=20truemm]{geometry}
\usepackage[utf8]{inputenc}
\usepackage{amsmath,amsthm,amssymb}
\usepackage{type1cm}
\newtheorem{theorem}{定理}[section]
%\newtheorem{proof}{証明{
\begin{document}
\maketitle
\section{第五章 離散フーリエ変換}
\subsection{サンプル値のフーリエ変換}
連続時間信号$x\left(t\right)$を間隔$T$でサンプリングした信号$x_{sampling} \left( t \right)$を、
\begin{equation}
x_{sampling} \left( t \right) = \sum_{n=0}^{ \infty } x \left( n\right) \delta \left( t-nT \right)
\end{equation}
と定義する。ここで、$\delta$関数は実数から実数への任意の連続関数$f \left( x \right)$について次の性質を持つ関数である。
\begin{equation}
\int_{- \infty }^{ \infty } f \left( x \right) \delta \left( x \right) dx = f \left( 0 \right)
\end{equation}
$x_{sampling} \left( t \right)$をフーリエ変換すると、
\begin{eqnarray}
	\mathfrak{I} \bigl\{ x_{sampling} \left( \omega \right) \bigr\} &=& X_{sampling} \left( t \right) \nonumber \\
	&=& \int _{ - \infty } ^{ \infty } \sum _{ n = 0 } ^{ \infty } x \left( n \right) \delta \left( t-nT \right) e ^ { -j \omega t } dt \nonumber \\
	&=& \sum _{ n = 0 } ^{ \infty } x \left( n \right) \int_{- \infty } ^{ \infty} \delta \left( t-nT \right) e ^{ -j \omega t } dt \nonumber \\
	&=& \sum _{ n = 0 } ^{ \infty } x \left( n \right) \int_{- \infty } ^{ \infty} \delta \left( t' \right) e ^{ -j \omega \left( t' + nT \right) } dt' \nonumber \\
	&=& \sum _{ n = 0 } ^{ \infty } x \left( n \right) e ^ { -j \omega nT }
\end{eqnarray}
となる。\\
ここで、$j$は虚数$ \left( j^2 = -1 \right)$、$\omega$は指数関数の角周波数である。\\
一般に信号は有限時間内に収まっているから、$x \left( n \right)$は$n$が$0$以上$N-1$以下の整数のときのみを考えればよいと考えることが出来る。よって、$n$が$0$より小さい、または$N$以上のとき$x \left( n \right) = 0$とすると、
\begin{equation}
\sum _{ n = 0 } ^{ \infty } x \left( n \right) e ^ { -j \omega nT } = \sum _{ n = 0 } ^{ N - 1 } x \left( n \right) e ^ { -j \omega nT }
\end{equation}
となる。\\
さらに、角周波数$\omega$について、波長が信号の時間幅である$NT$に等しくなる角周波数$\frac { 2 \pi } { TN }$を基本角周波数と呼び、$\omega$をその$k$倍($k$は$1$以上$N-1$以下の整数)で表すと、
\begin{equation}
\sum _{ n = 0 } ^{ N - 1 } x \left( n \right) e ^ { -j \omega nT } = \sum _{ n = 0 } ^{ N - 1 } x \left( n \right) e ^ { -j \frac { 2 \pi } { N } kn }
\end{equation}
となる。\\
簡単のため$W = e ^{ -j \frac { 2 \pi } { N } kn}$とおくと、
\begin{eqnarray}
	\mathfrak{I} \bigl\{ x_{sampling} \left( \omega \right) \bigr\} &=& X_{sampling} \left( t \right) \nonumber = X \left( k \right) \\
	&=& \sum _{ n = 0 } ^{ N - 1 } x \left( n \right) W ^{ kn }
\end{eqnarray}
となる。これを離散フーリエ変換(DFT:discrete Fourier transform)という。\\
逆フーリエ変換は、
\begin{equation}
	x \left( n \right) = \frac { 1 } { N } \sum _{ k = 0 } ^{ N - 1 } X \left( k \right) W ^ { -kn }
\end{equation}
で与えられる。\\
証明\\
$x \left( n \right)$を離散フーリエ変換し、さらに逆離散フーリエ変換したものを$f \left( m \right)$とすると、
\begin{eqnarray}
	f \left( m \right) &=& \frac{1}{N} \sum _{k = 0} ^{N - 1} \left( \sum _{n = 0} ^{N - 1} x \left( n \right) W ^{kn} \right) W ^{-km} \nonumber \\
	&=& \frac{1}{N} \sum _{k = 0} ^{N - 1} \sum _{n = 0} ^{N - 1} x \left( n \right) W ^{k \left( n - m \right) } \nonumber \\
	&=& \frac{1}{N} \sum _{n = 0} ^{N - 1} \sum _{k = 0} ^{N - 1} x \left( n \right) W ^{k \left( n - m \right) } \nonumber \\
	&=& \frac{1}{N} \sum _{n = 0} ^{N - 1} x \left( n \right) \sum _{k = 0} ^{N - 1} W ^{k \left( n - m \right) }
\end{eqnarray}
ここで、
\begin{equation}
\sum _{k = 0} ^{N - 1} W ^{k \left( n - m \right) } \nonumber
\end{equation}
について、\\
$n = m$のとき、
\begin{eqnarray}
	\sum _{k = 0} ^{N - 1} W ^{k \left( n - m \right) } &=& \sum _{k = 0} ^{N - 1} W ^0 \nonumber \\
	&=& \sum _{k = 0} ^{N - 1} 1 \nonumber \\
	&=& N \nonumber \\
\end{eqnarray}
$n \neq m$のとき、
\begin{eqnarray}
	\sum _{k = 0} ^{N - 1} W ^{k \left( n - m \right) } &=& \frac { 1 - W ^{ N \left( n - m \right) } } { 1 - W ^{ \left( n - m \right) } } \nonumber \\
	&=& \frac { 1 - \exp( -j \frac { 2 \pi } { N } N \left( n - m \right) ) } { 1 - W ^{ \left( n - m \right) } } \nonumber \\
	&=& \frac { 1 - \exp( -2 \pi j \left( n - m \right) ) } { 1 - W ^{ \left( n - m \right) } } \nonumber \\
	&=& \frac { 1 - 1 } { 1 - W ^{ \left( n - m \right) } } \nonumber \\
	&=& 0
\end{eqnarray}
よって、
\begin{eqnarray}
	\frac{1}{N} \sum _{n = 0} ^{N - 1} x \left( n \right) \sum _{k = 0} ^{N - 1} W ^{k \left( n - m \right) } &=& \frac { 1 } { N } N x \left( m \right) \nonumber \\
	&=& x \left( m \right)
\end{eqnarray}
よって、二つの関数$f$と$x$は等しく、逆離散フーリエ変換が離散フーリエ変換と逆変換の関係にあることが証明された。
\subsection{離散フーリエ変換の性質}
\subsubsection{線形性}
$a$と$b$を任意の実定数、$x_1 \left( n \right), x_2 \left( n \right)$の離散フーリエ変換をそれぞれ$X_1 \left( k \right), X_2 \left( k \right)$とする。
\begin{equation}
x \left( n \right) = ax_1 \left( n \right) + bx_2 \left( n \right)
\end{equation}
を満たす信号があるとするとその$DFT$は、
\begin{equation}
X \left( k \right) = aX_1 \left( k \right) + bX_2 \left( k \right)
\end{equation}
となる。\\
証明
\begin{eqnarray}
	X \left( k \right) &=& \sum _{n = 0} ^{N - 1} \left( ax_1 \left( n \right) + bx_2 \left( n \right) \right) W ^{kn} \nonumber \\
	&=& a \sum _{n = 0} ^{N -1} x_1 \left( n \right) W ^{kn} + b \sum _{n = 0} ^{N - 1} x_2 \left( n \right) W ^{kn} \nonumber \\
	&=& aX_1 \left( n \right) + bX_2 \left( n \right)
\end{eqnarray}
\subsubsection{周期性}
$r$を任意の整数とすると、
\begin{equation}
X \left( k + rN \right) = X \left( k \right)
\end{equation}
が成り立つ。つまり、$X \left( k \right)$は周期関数である。\\
証明
\begin{eqnarray}
	X \left( k + rN \right) &=& \sum _{n = 0} ^{N - 1} x \left( n \right) W ^{\left( k + rN \right) n} \nonumber \\
	&=& \sum _{n = 0} ^{N - 1} x \left( n \right) W ^{kn} W ^{rNn} \nonumber \\
	&=& \sum _{n = 0} ^{N - 1} x \left( n \right) W ^{kn} \exp(-j \frac{2 \pi}{N}rNn) \nonumber \\
	&=& \sum _{n = 0} ^{N - 1} x \left( n \right) W ^{kn} \exp(-2j \pi rn) \nonumber \\
	&=& \sum _{n = 0} ^{N - 1} x \left( n \right) W ^{kn} \left( \cos \left( -2 \pi rn \right) + j \sin \left(-2 \pi rn \right) \right) \nonumber \\
	&=& \sum _{n = 0} ^{N - 1} x \left( n \right) W ^{kn} \left( \cos 2 \pi rn - j \sin 2 \pi rn \right) \nonumber \\
	&=& \sum _{n = 0} ^{N - 1} x \left( n \right) W ^{kn} \nonumber \\
	&=& X \left( k \right)
\end{eqnarray}
\subsubsection{対称性}
\begin{equation}
X \left( N - k \right) = X^\ast \left( k \right)
\end{equation}
ただし、$X^\ast \left( k \right)$は$X \left( k \right)$の共役複素数である。\\
証明
\begin{eqnarray}
	X \left( N - k \right) &=& \sum _{n = 0} ^{N - 1} x \left( n \right) W ^{\left( N - k \right) n} \nonumber \\
	&=& \sum _{n = 0} ^{N - 1} x \left( n \right) W ^{Nn} W ^{-kn} \nonumber \\
	&=& \sum _{n = 0} ^{N - 1} x \left( n \right) W \exp \left(-j \frac{2 \pi}{N} Nn \right) \exp \left( -j \frac{2 \pi}{N} \left( -kn \right) \right) \nonumber \\
	&=& \sum _{n = 0} ^{N - 1} x \left( n \right) W \exp \left( -2 \pi nj \right) \exp \left( 2 \pi \frac{kn}{N} j \right) \nonumber \\
	&=& \sum _{n = 0} ^{N - 1} x \left( n \right) \left( \cos \left( -2 \pi n \right) + j \sin \left(-2 \pi n \right) \right) \left( \cos \left( 2 \pi \frac{kn}{N} \right) + j \sin \left( 2 \pi \frac{kn}{N} \right) \right) \nonumber \\
	&=& \sum _{n = 0} ^{N - 1} x \left( n \right) \left( \cos \left( 2 \pi \frac{kn}{N} \right) - j \sin \left( 2 \pi \frac{kn}{N} \right) \right) ^\ast \nonumber \\
	&=& \sum _{n = 0} ^{N - 1} x \left( n \right) \left( \cos \left( -2 \pi \frac{kn}{N} \right) + j \sin \left( -2 \pi \frac{kn}{N} \right) \right) ^\ast \nonumber \\
	&=& \sum _{n = 0} ^{N - 1} x \left( n \right) \left( \exp \left( -2 \pi j \frac{kn}{N} \right) \right) ^\ast \nonumber \\
	&=& \sum _{n = 0} ^{N - 1} x \left( n \right) \left( W ^{kn} \right) ^\ast \nonumber \\
	&=& \sum _{n = 0} ^{N - 1} \left( x \left( n \right) W ^{kn} \right) ^\ast \nonumber \\
	&=& \left( \sum _{n = 0} ^{N - 1} x \left( n \right) W ^{kn} \right) ^\ast \nonumber \\
	&=& X ^\ast \left( k \right)
\end{eqnarray}
\section{離散時間システム}
\subsection{サンプリング定理}
サンプリング信号
\begin{equation}
x_{sampling} \left( t \right) = \sum_{n=0}^{ \infty } x \left( n\right) \delta \left( t-nT \right)
\end{equation}
について、サンプリング周波数$ \frac { 1 } { T } $が低すぎると、元の信号$ x \left( t \right ) $に存在していた高い周波数成分が見えなくなり、見かけ上実際よりも低い周波数成分が現れているように見えてしまうことがある。この現象をエイリアシングという。\\
では、どの程度高い周波数でサンプリングをすればエイリアシングが起きないのだろうか?その答えは次のサンプリング定理によって示されている。
\begin{theorem}
信号$ x \left( t \right) $の最大周波数が$ f_{max} $のとき、$ \left( t \right) $を$ 2 f_{max} $以上のサンプリング周波数でサンプリングを行うと元の信号を復元することが出来る。
\end{theorem}
\begin{proof}
最大周波数$ f_{max} $の信号$ x \left( t \right) $のスペクトルを、虚数単位$ j \left( j^2 = -1 \right) $、角周波数$ \omega $を使って、
\begin{equation}
X \left( \omega \right) = \int _{ - \infty } ^\infty x \left( t \right) \exp \left( -j \omega t \right) dt
\end{equation}
とする。$x \left( t \right) $の最大周波数は$ f_{max} $より、最大角周波数は$ 2 \pi f_{max} $となる。よって$ X \left( \omega \right) $は$ | \omega | < 2 \pi f_{max} $の範囲に収まっているので$ X \left( \omega \right) $の波長を$ 4 \pi f_{max} $として複素フーリエ級数展開を適用すると、
\begin{eqnarray}
	X \left( \omega \right) & = & \sum _{ n = - \infty } ^\infty c_n \exp \left( - \frac { jn \omega } { 2 f_{max} } \right) \nonumber \\
	c_n & = & \frac { 1 } { 4 \pi f_{max} } \int _{ -2 \pi f_{max} } ^{ 2 \pi f_{max} } X \left( \omega \right) \exp \left( \frac { jn \omega } { 2 f_{max} } \right) d \omega
\end{eqnarray}
ここで、$ x \left( t \right) $の最大角周波数が$ 2 \pi f_{max} $であることを考慮すると、逆フーリエ変換の公式より、
\begin{equation}
x \left( t \right) = \frac { 1 } { 2 \pi } \int _{ -2 \pi f_{max} } ^{ 2 \pi f_{max} } X \left( \omega \right) \exp \left( j \omega t \right) d \omega
\end{equation}
より、
\begin{eqnarray}
	c_n & = & \frac { 1 } { 4 \pi f_{max} } \int _{ -2 \pi f_{max} } ^{ 2 \pi f_{max} } X \left( \omega \right) \exp \left( \frac { jn \omega } { 2 f_{max} } \right) d \omega \nonumber \\
	& = & \frac { 1 } { 2 f_{max} } \left( \frac { 1 } { 2 \pi } \int _{ -2 \pi f_{max} } ^{ 2 \pi f_{max} } X \left( \omega \right) \exp \left( \frac { jn \omega } { 2 f_{max} } \right) d \omega \right) \nonumber \\
	& = & \frac { 1 } { 2 f_{max} } x \left( \frac { n } { 2 f_{max} } \right) \nonumber \\
\end{eqnarray}
よって、サンプル点$ x \left( \frac { n } { 2 f_{max} } \right) \left( n \in \mathbb{Z} \right) $が与えられると、$ x \left( t \right) $のスペクトル$ X \left( \omega \right) $の複素フーリエ級数展開の係数が定まり、$ X \left( \omega \right) $が求められ、それに逆フーリエ変換を適用するとこで最終的に$ x \left( t \right) $を復元することが出来る。
\end{proof}
ここで、実際に$ x_n = x \left( \frac { n } { 2 f_{max} } \right) \left( n \in \mathbb{Z} \right) $を用いて$ x \left( t \right) $を表現してみる。
\begin{eqnarray}
	x \left( t \right) & = & \frac { 1 } { 2 \pi } \int _{ -2 \pi f_{max} } ^{ 2 \pi f_{max} } X \left( \omega \right) \exp \left( j \omega t \right) d \omega \nonumber \\
	& = & \frac { 1 } { 2 \pi } \int _{ -2 \pi f_{max} } ^{ 2 \pi f_{max} } \sum _{ n = - \infty } ^\infty c_n \exp \left( - \frac { jn \omega } { 2 f_{max} } \right) \exp \left( j \omega t \right) d \omega \nonumber \\
	& = & \frac { 1 } { 2 \pi } \int _{ -2 \pi f_{max} } ^{ 2 \pi f_{max} } \sum _{ n = - \infty } ^\infty \frac { 1 } { 2 f_{max} } x \left( \frac { n } { 2 f_{max} } \right) \exp \left( - \frac { jn \omega } { 2 f_{max} } \right) \exp \left( j \omega t \right) d \omega \nonumber \\
	& = & \frac { 1 } { 2 \pi } \int _{ -2 \pi f_{max} } ^{ 2 \pi f_{max} } \sum _{ n = - \infty } ^\infty \frac { x_n } { 2 f_{max} } \exp \left( - \frac { jn \omega } { 2 f_{max} } \right) \exp \left( j \omega t \right) d \omega \nonumber \\
	& = & \frac { 1 } { 2 \pi } \int _{ -2 \pi f_{max} } ^{ 2 \pi f_{max} } \sum _{ n = - \infty } ^\infty \frac { x_n } { 2 f_{max} } \exp \left( j \omega t - \frac { jn \omega } { 2 f_{max} } \right) d \omega \nonumber \\
	& = & \frac { 1 } { 2 \pi } \int _{ -2 \pi f_{max} } ^{ 2 \pi f_{max} } \frac { 1 } { 2 f_{max} } \sum _{ n = - \infty } ^\infty x_n \exp \left( j \omega \left( t - \frac { n } { 2 f_{max} } \right) \right) d \omega \nonumber \\
	& = & \frac { 1 } { 4 \pi f_{max} } \sum _{ n = - \infty } ^\infty x_n \int _{ -2 \pi f_{max} } ^{ 2 \pi f_{max} } \exp \left( j \omega \left( t - \frac { n } { 2 f_{max} } \right) \right) d \omega \nonumber \\
	& = & \frac { 1 } { 4 \pi f_{max} } \sum _{ n = - \infty } ^\infty x_n \left[ \frac { 1 } { j \left( t - \frac { n } { 2 f_{max} } \right) } \exp \left( j \omega \left( t - \frac { n } { 2 f_{max} } \right) \right) \right] _{ \omega = -2 \pi f_{max} } ^{ \omega = 2 \pi f_{max} } \nonumber \\
	& = & \frac { 1 } { 4 \pi f_{max} } \sum _{ n = - \infty } ^\infty x_n \left( -j \frac { 2 f_{max} } { 2tf_{max} - n } \right) \left[ \exp \left( j \omega \left( t - \frac { n } { 2 f_{max} } \right) \right) \right] _{ \omega = -2 \pi f_{max} } ^{ \omega = 2 \pi f_{max} } \nonumber \\
	& = & \frac { 1 } { 4 \pi f_{max} } \sum _{ n = - \infty } ^\infty x_n \left( -j \frac { 2 f_{max} } { 2tf_{max} - n } \right) \left[ \cos \left( \omega \left( t - \frac { n } { 2 f_{max} } \right) \right) + j \sin \left( \omega \left( t - \frac { n } { 2 f_{max} } \right) \right) \right] _{ \omega = -2 \pi f_{max} } ^{ \omega = 2 \pi f_{max} } \nonumber \\
	& = & \frac { 1 } { 4 \pi f_{max} } \sum _{ n = - \infty } ^\infty x_n \left( -j \frac { 2 f_{max} } { 2tf_{max} - n } \right) \left( 2 j \sin \left( 2 \pi f_{max} t - n \pi \right) \right) \nonumber \\
	& = & \frac { 1 } { \pi } \sum _{ n = - \infty } ^\infty \frac { x_n } { 2tf_{max} - n } \sin \left( 2 \pi f_{max} t - n \pi \right)
\end{eqnarray}
\end{document}
