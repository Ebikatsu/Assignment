\documentclass[a4paper]{jsarticle}
\usepackage[top=20truemm,bottom=20truemm,left=20truemm,right=20truemm]{geometry}
\usepackage{amsmath,amssymb}
\usepackage{comment}
\usepackage{type1cm}
\title{フーリエ級数展開}
\author{伊藤 太清}
\begin{document}
\maketitle
\section{フーリエ級数展開とは}
関数$f: \mathbb{R} \longrightarrow \mathbb{R}$がディリクレの条件を満たし、かつ周期$T$を持つ、すなわち$\forall x \in \mathbb{R} , f \left( x + T \right) = f \left( x \right)$を満たすとき、$f \left( x \right)$は、
\begin{equation}
\label{FourierSeriesFomula}
f \left( x \right) = \frac { a_0 } { 2 } + \sum _{ n = 1} ^{ \infty } \left( a_n \cos( \frac { 2 \pi nx } { T } ) + b_n \sin( \frac { 2 \pi nx } { T } ) \right)
\end{equation}
の形であらわすことが出来る。またこの形で表したとき、係数$a_n,b_n$は、
\begin{eqnarray}
a_n = \frac { 2 } { T } \int _0 ^T f \left( x \right) \cos( \frac { 2 \pi nx } { T } ) dx \\
b_n = \frac { 2 } { T } \int _0 ^T f \left( x \right) \sin( \frac { 2 \pi nx } { T } ) dx
\end{eqnarray}
となる。
\section{係数の導出}
フーリエ級数展開が成り立つ関数$f: \mathbb{R} \longrightarrow \mathbb{R}$について、
\begin{equation}
f \left( x \right) = \frac { a_0 } { 2 } + \sum _{ n = 1} ^{ \infty } \left( a_n \cos( \frac { 2 \pi nx } { T } ) + b_n \sin( \frac { 2 \pi nx } { T } ) \right)
\end{equation}
まず、$a_0$を導出する。\\
式\ref{FourierSeriesFomula}の両辺を$x \in \left[ 0,T \right]$で積分して、
\begin{eqnarray}
	\int _0 ^T f \left( x \right) dx &=& \int _0 ^T \left( \frac { a_0 } { 2 } + \sum _{ n = 1} ^{ \infty } \left( a_n \cos( \frac { 2 \pi nx } { T } ) + b_n \sin( \frac { 2 \pi nx } { T } ) \right) \right) dx \nonumber \\
	&=& \left[ \frac { a_0 } { 2 } + \sum _{ n = 1} ^\infty \left( a_n \frac { T } { 2 \pi n } \sin( \frac { 2 \pi nx } { T } ) - b_n \frac { T } { 2 \pi n } \cos( \frac { 2 \pi nx } { T } ) \right) \right] _{ x = 0 } ^{ x = T } \nonumber \\
	&=& \left( \frac { a_0T } { 2 } + \sum _{ n = 1 } ^\infty \left( \frac { a_0T } { 2 \pi n } \sin( 2 \pi n ) - \frac { b_nT } { 2 \pi n } \cos( 2 \pi n) \right) \right) - \left( \sum _{ m = 1 } ^\infty \left( - \frac { b_mT } { 2 \pi m } \right) \right) \nonumber \\
	&=& \frac { a_0T } { 2 } + \sum _{ n = 1 } ^\infty \left( - \frac { b_nT } { 2 \pi n } \right) - \sum _{ m = 1 } ^\infty \left( - \frac { b_mT } { 2 \pi m } \right) \nonumber \\
	&=& \frac { a_0T } { 2 }
\end{eqnarray}
よって、
\begin{equation}
a_0 = \frac { 2 } { T } \int _0 ^T f \left( x \right) dx
\end{equation}
次に、$a_m \left( m \in \mathbb{N} \right)$を導出する。
式\ref{FourierSeriesFomula}の両辺に$\cos(\frac { 2 \pi mx } { T })$をかけると、
\begin{equation}
	f \left( x \right) \cos(\frac { 2 \pi mx } { T }) = \sum _{ n = 1 } ^\infty \left( a_n \cos(\frac { 2 \pi nx } { T }) \cos(\frac { 2 \pi mx } { T }) + b_n \sin(\frac { 2 \pi nx } { T }) \cos(\frac { 2 \pi mx } { T }) \right)
\end{equation}
ここで、$\cos(\alpha x) \cos(\beta x)$と、$\sin(\alpha x) \cos(\beta x)$について考えてみる。\\
加法定理より、
\begin{eqnarray}
	\cos( \alpha x + \beta x ) = \cos( \alpha x ) \cos( \beta x ) - \sin( \alpha x ) \sin( \beta x ) \nonumber \\
	\cos( \alpha x - \beta x ) = \cos( \alpha x ) \cos( \beta x ) + \sin( \alpha x ) \sin( \beta x ) \nonumber
\end{eqnarray}
上の二つの式を足し合わせると、
\begin{equation}
	\cos( \alpha x + \beta x ) + \cos( \alpha x - \beta x ) = 2 \cos( \alpha x ) \cos( \beta x )
\end{equation}
よって、
\begin{equation}
	\frac { 1 } { 2 } \cos( \alpha x + \beta x ) + \frac { 1 } { 2 } \cos( \alpha x - \beta x ) = \cos( \alpha x ) \cos( \beta x )
\end{equation}
同様に、
\begin{eqnarray}
	\sin( \alpha x + \beta x ) = \sin( \alpha x ) \cos( \beta x ) + \cos( \alpha x ) \sin( \beta x ) \nonumber \\
	\sin( \alpha x - \beta x ) = \sin( \alpha x ) \cos( \beta x ) - \cos( \alpha x ) \sin( \beta x ) \nonumber
\end{eqnarray}
上の二つの式を足し合わせると、
\begin{equation}
	\sin( \alpha x + \beta x ) + \sin( \alpha x - \beta x ) = 2 \sin( \alpha x ) \cos( \beta x )
\end{equation}
よって、
\begin{equation}
	\frac { 1 } { 2 } \sin( \alpha x + \beta x ) + \frac { 1 } { 2 } \sin( \alpha x - \beta x ) = \sin( \alpha x ) \cos( \beta x )
\end{equation}
これらを適用すると、
\begin{eqnarray}
	&f& \left( x \right) \cos \frac { 2 \pi mx } { T } \nonumber \\
	&=& \frac { a_0 } { 2 } \cos \frac { 2 \pi mx } { T } \nonumber \\
	&+& \sum _{ n = 1} ^\infty \left( \frac { a_n } { 2 } \cos \frac { 2 \pi \left( n + m \right) x } { T } + \frac { a_n } { 2 } \cos \frac { 2 \pi \left( n - m \right) x } { T } + \frac { b_n } { 2 } \sin \frac { 2 \pi \left( n + m \right) x } { T } + \frac { b_n } { 2 } \sin \frac { 2 \pi \left( n - m \right) x } { T } \right)
\end{eqnarray}
両辺$x \in \left[ 0,T \right]$で積分すると、
\begin{eqnarray}
	&{\int _0 ^T}& f \left( x \right) \cos \frac { 2 \pi mx } { T } dx \nonumber \\
	&=& \frac { a_0 } { 2 } \int _0 ^T \cos \frac { 2 \pi mx } { T } dx \nonumber \\
	&+& \sum _{ n = 1 } ^\infty \left( \frac { a_n } { 2 } \int _0 ^T \cos \frac { 2 \pi \left( n + m \right) x } { T }  dx \right. \nonumber \\
	&+& \frac { a_n } { 2 } \int _0 ^T \cos \frac { 2 \pi \left( n - m \right) x } { T } dx \nonumber \\
	&+& \frac { b_n } { 2 } \int _0 ^T \sin \frac { 2 \pi \left( n + m \right) x } { T } dx \nonumber \\
	&+& \left. \frac { b_n } { 2 } \int _0 ^T \sin \frac { 2 \pi \left( n - m \right) x } { T } dx \right) \nonumber \\
	&=& \frac { a_0 } { 2 } \left[ \frac { T } { 2 \pi m } \sin \frac { 2 \pi mx } { T } \right] _{ x = 0 } ^{ x + T } \nonumber \\
	&+& \sum _{ n = 1 } ^\infty \left( \frac { a_n } { 2 } \left[ \frac { T } { 2 \pi \left( n + m \right) } \sin \frac { 2 \pi \left( n + m \right) x } { T } \right] _{ x = 0 } ^{ x = T } \right. \nonumber \\
	&+& \frac { a_n } { 2 } \left[ \frac { T } { 2 \pi \left( n - m \right) } \sin \frac { 2 \pi \left( n - m \right) x } { T } \right] _{ x = 0 } ^{ x = T } \nonumber \\
	&+& \frac { b_n } { 2 } \left[ \frac { T } { 2 \pi \left( n + m \right) } \cos \frac { 2 \pi \left( n + m \right) x } { T } \right] _{ x = 0 } ^{ x = T } \nonumber \\
	&+& \left. \frac { b_n } { 2 } \left[ \frac { T } { 2 \pi \left( n - m \right) } \cos \frac { 2 \pi \left( n - m \right) x } { T } \right] _{ x = 0 } ^{ x = T }\right) 
\end{eqnarray}
ここで、最初の項は$0$、シグマの中の四つの項についても、$n \neq m$のとき、いずれも$0$になるから、
\begin{eqnarray}
	&{\int _0 ^T}& f \left( x \right) \cos \frac { 2 \pi mx } { T } dx \nonumber \\
	&=& \int _0 ^T \left( \frac { a_0 } { 2 } \cos \frac { 2 \pi mx } { T } \right. \nonumber \\
	&+& \left. \sum _{ n = 1} ^\infty \left( \frac { a_n } { 2 } \cos \frac { 2 \pi \left( n + m \right) x } { T } + \frac { a_n } { 2 } \cos \frac { 2 \pi \left( n - m \right) x } { T } + \frac { b_n } { 2 } \sin \frac { 2 \pi \left( n + m \right) x } { T } + \frac { b_n } { 2 } \sin \frac { 2 \pi \left( n - m \right) x } { T } \right) \right) dx \nonumber \\
	&=& \int _0 ^T \left( a_m \cos ^2 \frac { 2 \pi mx } { T } + b_m \sin \frac { 2 \pi mx } { T } \cos \frac { 2 \pi mx } { T } \right) dx \nonumber \\
	&=& \int _0 ^T \left( a_m \left( \frac { 1 } { 2 } \left( 1 + \cos \frac { 4 \pi mx } { T } \right) \right) + \frac { b_m } { 2 } \sin \frac { 4 \pi mx } { T } \right) dx \nonumber \\
	&=& \int _0 ^T \left( \frac { a_m } { 2 } + \frac { a_m } { 2 } \cos \frac { 4 \pi mx } { T } + \frac { b_m } { 2 } \sin \frac { 4 \pi mx } { T } \right) dx \nonumber \\
	&=& \int _0 ^T \frac { a_m } { 2 } dx + \frac { a_m } { 2 } \int _0 ^T \cos \frac { 4 \pi mx } { T } dx + \frac { b_m } { 2 } \int _0 ^T \sin \frac { 4 \pi mx } { T } dx \nonumber \\
	&=& \left[ \frac { a_m } { 2 } x \right] _{ x = 0 } ^{ x = T } + \frac { a_m } { 2 } \left[ \frac { T } { 4 \pi m } \sin \frac { 4 \pi mx } { T } \right] _{ x = 0 } ^{ x = T } + \frac { b_m } { 2 } \left[ - \frac { T } { 4 \pi m } \cos \frac { 4 \pi mx } { T } \right] _{ x = 0 } ^{ x = T } \nonumber \\
	&=& \frac { a_mT } { 2 }
\end{eqnarray}
よって、
\begin{equation}
a_m = \frac { 2 } { T } \int _0 ^T f \left( x \right) \cos \frac { 2 \pi mx } { T } dx
\end{equation}
同様に、式\ref{FourierSeriesFomula}の両辺に$\sin \frac { 2 \pi mx } { T }$をかけ、$x \in \left[ 0,T \right]$で積分することによって、
\begin{equation}
b_m = \frac { 2 } { T } \int _0 ^T f \left( x \right) \sin \frac { 2 \pi mx } { T } dx
\end{equation}
を得ることが出来る。
\section{ディリクレ核}
周期$T$の関数$ f : \mathbb{R} \to \mathbb{R} $について、$ S_N \left( x \right) $を、$ f \left( x \right) $のフーリエ級数展開の$ n = N $までの和、すなわち
\begin{equation}
S_N \left( x \right) = \frac { 1 } { T } \int _0 ^T f \left( y \right) dy + \sum _{ n = 1 } ^N \left( \frac { 2 } { T } \cos \left( \frac { 2 \pi nx } { T } \right) \int _0 ^T f \left( y \right) \cos \left( \frac { 2 \pi ny } { T } \right) dy + \frac { 2 } { T } \sin \left( \frac { 2 \pi nx } { T } \right) \int _0 ^T f \left( y \right) \sin \left( \frac { 2 \pi ny } { T } \right) dy \right)
\end{equation}
と定義すると、$f$がフーリエ展開可能であることは、
\begin{equation}
\lim _{ N \to \infty } S_N \left( x \right) - f \left( x \right) = 0
\end{equation}
と同値である。
\begin{eqnarray}
	S_N \left( x \right) & = & \frac { 1 } { T } \int _0 ^T f \left( y \right) dy + \sum _{ n = 1 } ^N \left( \frac { 2 } { T } \cos \left( \frac { 2 \pi nx } { T } \right) \int _0 ^T f \left( y \right) \cos \left( \frac { 2 \pi ny } { T } \right) dy + \frac { 2 } { T } \sin \left( \frac { 2 \pi nx } { T } \right) \int _0 ^T f \left( y \right) \sin \left( \frac { 2 \pi ny } { T } \right) dy \right) \nonumber \\
	& = & \int _0 ^T \left( \frac { 1 } { T } f \left( y \right) + \sum _{ n = 1 } ^N \left( \frac { 2 } { T } \cos \left( \frac { 2 \pi nx } { T } \right) f \left( y \right) \cos \left( \frac { 2 \pi ny } { T } \right) + \frac { 2 } { T } \sin \left( \frac { 2 \pi nx } { T } \right) f \left( y \right) \sin \left( \frac { 2 \pi ny } { T } \right) \right) \right) dy \nonumber \\
	& = & \int _0 ^T f \left( y \right) \left( \frac { 1 } { T } + \sum _{ n = 1 } ^N \left( \frac { 2 } { T } \cos \left( \frac { 2 \pi nx } { T } \right) \cos \left( \frac { 2 \pi ny } { T } \right) + \frac { 2 } { T } \sin \left( \frac { 2 \pi nx } { T } \right) \sin \left( \frac { 2 \pi ny } { T } \right) \right) \right) dy \nonumber \\
	& = & \frac { 2 } { T } \int _0 ^T f \left( y \right) \left( \frac { 1 } { 2 } + \sum _{ n = 1 } ^N \left( \cos \left( \frac { 2 \pi nx } { T } \right) \cos \left( \frac { 2 \pi ny } { T } \right) + \sin \left( \frac { 2 \pi nx } { T } \right) \sin \left( \frac { 2 \pi ny } { T } \right) \right) \right) dy \nonumber \\
	& = & \frac { 2 } { T } \int _0 ^T f \left( y \right) \left( \frac { 1 } { 2 } + \sum _{ n = 1 } ^N \cos \left( \frac { 2 \pi n \left( x - y \right) } { T } \right) \right) dy 
\end{eqnarray}
ここで、$ z = y - x $として、
\begin{eqnarray}
	S_N \left( x \right) & = & \frac { 2 } { T } \int _0 ^T f \left( y \right) \left( \frac { 1 } { 2 } + \sum _{ n = 1 } ^N \cos \left( \frac { 2 \pi n \left( x - y \right) } { T } \right) \right) dy \nonumber \\
	& = & \frac { 2 } { T } \int _{ -x } ^{ T - x } f \left( x + z \right) \left( \frac { 1 } { 2 } + \sum _{ n = 1 } ^N \cos \left( \frac { 2 \pi nz ) } { T } \right) \right) dz \nonumber \\
	& = & \frac { 2 } { T } \int _0 ^T f \left( x + z \right) \left( \frac { 1 } { 2 } + \sum _{ n = 1 } ^N \cos \left( \frac { 2 \pi nz ) } { T } \right) \right) dz
\end{eqnarray}
ここで、ディリクレ核
\begin{equation}
D_N \left( z \right) = \frac { 1 } { 2 } + \sum _{ n = 1 } ^N \cos \left( \frac { 2 \pi nz } { T } \right)
\end{equation}
について、
\begin{eqnarray}
	2 \sin \left( \frac { \pi z } { T } \right) D_N \left( z \right) & = & \sin \left( \frac { \pi z } { T } \right) + \sum _{ n = 1 } ^N 2 \sin \left( \frac { \pi z } { T } \right) \cos \left( \frac { 2 \pi nz } { T } \right) \nonumber \\
	& = & \sin \left( \frac { \pi z } { T } \right) + \sum _{ n = 1 } ^N 2 \left( \sin \left( \frac { \left( 2n + 1 \right) \pi z } { T } \right) + \sin \left( \frac { \left( 1 - 2n \right) \pi z } { T } \right) \right) \nonumber \\
	& = & \sin \left( \frac { \pi z } { T } \right) + \sum _{ n = 1 } ^N 2 \left( \sin \left( \frac { \left( 2n + 1 \right) \pi z } { T } \right) - \sin \left( \frac { \left( 2n - 1 \right) \pi z } { T } \right) \right) \nonumber \\
	& = & \sin \left( \frac { \pi z } { T } \right) - \sin \left( \frac { \pi z } { T } \right) + \sin \left( \frac { \left( 2N + 1 \right) \pi z } { T } \right) \nonumber \\
	& = & \sin \left( \frac { \left( 2N + 1 \right) \pi z } { T } \right) 
\end{eqnarray}
よって、
\begin{equation}
D_N \left( z \right) = \frac { \sin \left( \frac { \left( 2N + 1 \right) \pi z } { T } \right) } { 2 \sin \left( \frac { \pi z } { T } \right) }
\end{equation}
よって、
\begin{eqnarray}
	S_N \left( x \right) & = & \frac { 2 } { T } \int _0 ^T f \left( y \right) \left( \frac { 1 } { 2 } + \sum _{ n = 1 } ^N \cos \left( \frac { 2 \pi n \left( x - y \right) } { T } \right) \right) dy \nonumber \\
	S_N \left( x \right) & = & \frac { 2 } { T } \int _0 ^T f \left( y \right) \frac { \sin \left( \frac { \left( 2N + 1 \right) \pi z } { T } \right) } { 2 \sin \left( \frac { \pi z } { T } \right) }
dy
\end{eqnarray}
\section{複素フーリエ級数展開}
また、$ f \left( x \right) $のフーリエ展開は複素数を用いて以下のように表すこともできる。
\begin{equation}
\label{ComplexFourierSeriesFomula}
f \left( x \right) = \sum _{ n = - \infty } ^\infty c_n \exp \left( \frac { 2 \pi inx } { T } \right)
\end{equation}
この時係数$ c_n $は、
\begin{equation}
c_n = \frac { 1 } { T } \int _0 ^T f \left( x \right) \exp \left( - \frac { 2 \pi inx } { T } \right) dx
\end{equation}
となる。
\section{係数の導出}
式\ref{ComplexFourierSeriesFomula}の両辺に$ \exp \left( - \frac { 2 \pi imx } { T } \right) $をかけて$ x \in \left[ 0,T \right] $で積分すると、
\begin{eqnarray}
	& \int _0 ^T & f \left( x \right) \exp \left( - \frac { 2 \pi inx } { T } \right) dx \nonumber \\
	&=& \int _0 ^T \sum _{ n = - \infty } ^\infty c_n \exp \left( \frac { 2 \pi inx } { T } \right) \exp \left( - \frac { 2 \pi imx } { T } \right) dx \nonumber \\
	&=& \int _0 ^T \sum _{ n = - \infty } ^\infty c_n \exp \left( \frac { 2 \pi i \left( n - m \right) x } { T } \right) dx \nonumber \\
	&=& \sum _{ n = - \infty } ^\infty c_n \int _0 ^T \exp \left( \frac { 2 \pi i \left( n - m \right) x } { T } \right) dx \nonumber \\
	&=& \sum _{ n = - \infty } ^\infty c_n \left[ \frac { T } { 2 \pi i \left( n - m \right) } \exp \left( \frac { 2 \pi i \left( n - m \right) x } { T } \right) \right] _{ x = 0 } ^{ x = T } \nonumber \\
	&=& \sum _{ n = - \infty } ^\infty \frac { -c_nTi } { 2 \pi \left( n - m \right) } \left[ \exp \left( \frac { 2 \pi i \left( n - m \right) x } { T } \right) \right] _{ x = 0 } ^{ x = T } \nonumber \\
	&=& \sum _{ n = - \infty } ^\infty \frac { -c_nTi } { 2 \pi \left( n - m \right) } \left( \exp \left( 2 \pi i \left( n - m \right) \right) - 1 \right) \nonumber \\
	&=& \sum _{ n = - \infty } ^\infty \frac { -c_nTi } { 2 \pi \left( n - m \right) } \left( \cos \left( 2 \pi \left( n - m \right) \right) + i \sin \left( 2 \pi \left( n - m \right) \right) - 1 \right)
\end{eqnarray}
ここで、$ n,m \in \mathbb{Z} $より、
\begin{eqnarray}
	& \int _0 ^T & f \left( x \right) \exp \left( - \frac { 2 \pi inx } { T } \right) dx \nonumber \\
	&=& \sum _{ n = - \infty } ^\infty \frac { -c_nTi } { 2 \pi \left( n - m \right) } \left( \cos \left( 2 \pi \left( n - m \right) \right) + i \sin \left( 2 \pi \left( n - m \right) \right) - 1 \right) \nonumber \\
	&=& \sum _{ n = - \infty } ^\infty \frac { -c_nTi } { 2 \pi \left( n - m \right) } \left( \cos \left( 2 \pi \left( n - m \right) \right) + i \sin \left( 2 \pi \left( n - m \right) \right) - 1 \right) \nonumber \\
	&=& \sum _{ n = - \infty } ^\infty \frac { -c_nTi } { 2 \pi \left( n - m \right) } \cdot 0 \nonumber \\
\end{eqnarray}
よって$ n \neq m $の時の項は全て$0$であるから、
\begin{eqnarray}
	& \int _0 ^T & f \left( x \right) \exp \left( - \frac { 2 \pi inx } { T } \right) dx \nonumber \\
	&=& \int _0 ^T \sum _{ n = - \infty } ^\infty c_n \exp \left( \frac { 2 \pi i \left( n - m \right) x } { T } \right) dx \nonumber \\
	&=& \int _0 ^T c_m \exp \left( \frac { 2 \pi i \left( m - m \right) x } { T } \right) dx \nonumber \\
	&=& \int _0 ^T c_m \exp \left( 0 \right) dx \nonumber \\
	&=& \int _0 ^T c_m dx \nonumber \\
	&=& \left[ c_m x \right] _{ x = 0 } ^{ x = T } \nonumber \\
	&=& c_m T
\end{eqnarray}
よって、
\begin{equation}
c_n = \frac { 1 } { T } \int _0 ^T f \left( x \right) \exp \left( - \frac { 2 \pi inx } { T } \right) dx
\end{equation}
\section{実数のみを使ったフーリエ級数展開から複素フーリエ級数展開を導出する}
実数のみを使ったフーリエ級数展開から複素フーリエ級数展開を導出することもできる。\\
オイラーの公式
\begin{equation}
\exp \left( ix \right) = \cos x + i \sin x
\end{equation}
より、
\begin{eqnarray}
	\exp \left( i \theta \right) + \exp \left( -i \theta \right) &=& \cos \theta + i \sin \theta + \cos \left( - \theta \right) + i \sin \left( - \theta \right) \nonumber \\
	&=& \cos \theta + i \sin \theta + cos \theta - i \sin \theta \nonumber \\
	&=& 2 \cos \theta
\end{eqnarray}
よって、
\begin{equation}
\label{CosToExp}
\cos \theta = \frac { \exp \left( i \theta \right) + \exp \left( -i \theta \right) } { 2 }
\end{equation}
また、
\begin{eqnarray}
	\exp \left( i \theta \right) - \exp \left( -i \theta \right) &=& \cos \theta + i \sin \theta - \cos \left( - \theta \right) - i \sin \left( - \theta \right) \nonumber \\
	&=& \cos \theta + i \sin \theta - cos \theta + i \sin \theta \nonumber \\
	&=& 2i \sin \theta
\end{eqnarray}
よって、
\begin{equation}
\label{SinToExp}
\sin \theta = \frac { \exp \left( i \theta \right) - \exp \left( -i \theta \right) } { 2i }
\end{equation}
式\ref{CosToExp},\ref{SinToExp}を式\ref{FourierSeriesFomula}に代入すると、
\begin{eqnarray}
	f \left( x \right) &=& \frac { a_0 } { 2 } + \sum _{ n = 1 } ^\infty \left( \frac { a_n } { 2 } \left( \exp \left( \frac { 2 \pi inx } { T } \right) + \exp \left( - \frac { 2 \pi inx } { T } \right) \right) + \frac { b_n } { 2 } \left( \exp \left( \frac { 2 \pi inx } { T } \right) - \exp \left( - \frac { 2 \pi inx } { T } \right) \right) \right) \nonumber \\
	&=& \frac { a_0 } { 2 } + \sum _{ n = 1 } ^\infty \left( \frac { a_n } { 2 } \exp \left( \frac { 2 \pi inx } { T } \right) + \frac { a_n } { 2 } \exp \left( - \frac { 2 \pi inx } { T } \right) + \frac { b_n } { 2 } \exp \left( \frac { 2 \pi inx } { T } \right) - \frac { b_n } { 2 } \exp \left( - \frac { 2 \pi inx } { T } \right) \right) \nonumber \\
	&=& \frac { a_0 } { 2 } + \sum _{ n = 1 } ^\infty \left( \frac { a_n - b_n i } { 2 } \exp \left( \frac { 2 \pi inx } { T } \right) + \frac { a_n + b_n i } { 2 } \exp \left( - \frac { 2 \pi inx } { T } \right) \right) \nonumber \\
	&=& \frac { a_0 } { 2 } \exp \left( \frac { 2 \pi i \cdot 0 \cdot x } { T } \right) + \sum _{ n = 1 } ^\infty \frac { a_n - b_n i } { 2 } \exp \left( \frac { 2 \pi inx } { T } \right) + \sum _{ n = -1 } ^{ - \infty } \frac { a_{-n} + b_{-n} i } { 2 } \exp \left( \frac { 2 \pi inx } { T } \right) \nonumber \\
	&=& \sum _{ n = - \infty } ^\infty c_n \exp \left( \frac { 2 \pi inx } { T } \right)
\end{eqnarray}
\section{係数の導出}
$c_0$について、
\begin{eqnarray}
	& c_0 & - \left[ \frac { 1 } { T } \int _0 ^T f \left( x \right) \exp \left( - \frac { 2 \pi inx } { T } \right) \right] _{ n = 0 } \nonumber \\
	&=& \frac { a_0 } { 2 } - \frac { 1 } { T } \int _0 ^T f \left( x \right) dx \nonumber \\
	&=& \frac { 1 } { 2 } \left[ \frac { 2 } { T } \int _0 ^T f \left( x \right) \cos \left( \frac { 2 \pi nx } { T } \right) dx \right] _{ n = 0 } - \frac { 1 } { T } \int _0 ^T f \left( x \right) dx \nonumber \\
	&=& \frac { 1 } { T } \int _0 ^T f \left( x \right) dx - \frac { 1 } { T } \int _0 ^T f \left( x \right) dx \nonumber \\
	&=& 0
\end{eqnarray}
よって、
\begin{equation}
	c_0 = \left[ \frac { 1 } { T } \int _0 ^T f \left( x \right) \exp \left( - \frac { 2 \pi inx } { T } \right) \right] _{ n = 0 }
\end{equation}
$ c=n \left( n > 0 \right) $について、
\begin{eqnarray}
	& c_n & - \frac { 1 } { T } \int _0 ^T f \left( x \right) \exp \left( - \frac { 2 \pi inx } { T } \right) dx \nonumber \\
	&=& \frac { a_n - b_n i } { 2 } - \frac { 1 } { T } \int _0 ^T f \left( x \right) \exp \left( - \frac { 2 \pi inx } { T } \right) dx \nonumber \\
	&=& \frac { 1 } { 2 } \frac { 2 } { T } \int _0 ^T f \left( x \right) \cos \left( \frac { 2 \pi nx } { T } \right) dx - \frac { 1 } { 2 } i \frac { 2 } { T } \int _0 ^T f \left( x \right) \sin \left( \frac { 2 \pi nx } { T } \right) dx - \frac { 1 } { T } \int _0 ^T f \left( x \right) \exp \left( - \frac { 2 \pi inx } { T } \right) dx \nonumber \\
	&=& \frac { 1 } { T } \int _0 ^T f \left( x \right) \cos \left( \frac { 2 \pi nx } { T } \right) dx - \frac { 1 } { T } i \int _0 ^T f \left( x \right) \sin \left( \frac { 2 \pi nx } { T } \right) dx - \frac { 1 } { T } \int _0 ^T f \left( x \right) \left( \cos \left( - \frac { 2 \pi nx } { T } \right) + i \sin \left( - \frac { 2 \pi nx } { T } \right) \right) dx \nonumber \\
	&=& \frac { 1 } { T } \int _0 ^T f \left( x \right) \cos \left( \frac { 2 \pi nx } { T } \right) dx - \frac { 1 } { T } i \int _0 ^T f \left( x \right) \sin \left( \frac { 2 \pi nx } { T } \right) dx - \frac { 1 } { T } \int _0 ^T f \left( x \right) \left( \cos \left( \frac { 2 \pi nx } { T } \right) - i \sin \left( \frac { 2 \pi nx } { T } \right) \right) dx \nonumber \\
	&=& \frac { 1 } { T } \int _0 ^T f \left( x \right) \cos \left( \frac { 2 \pi nx } { T } \right) dx - \frac { 1 } { T } i \int _0 ^T f \left( x \right) \sin \left( \frac { 2 \pi nx } { T } \right) dx - \frac { 1 } { T } \int _0 ^T f \left( x \right) \cos \left( \frac { 2 \pi nx } { T } \right) dx + \frac { 1 } { T } i \int _0 ^T f \left( x \right) \sin \left( \frac { 2 \pi nx } { T } \right) dx \nonumber \\
	&=& 0
\end{eqnarray}
よって$ n > 0 $のとき、
\begin{equation}
c_n = \frac { 1 } { T } \int _0 ^T f \left( x \right) \exp \left( - \frac { 2 \pi inx } { T } \right) dx
\end{equation}
$ n < 0 $のときも同様に証明できる。
\begin{comment}
\section{フーリエ級数展開からフーリエ変換へ}
複素フーリエ級数展開の係数
\begin{equation}
c_n = \frac { 1 } { T } \int _0 ^T f \left( x \right) \exp \left( - \frac { 2 \pi inx } { T } \right) dx
\end{equation}
について、両辺に$T$をかけて、
\begin{eqnarray}
	c_n T & = & \int _0 ^T f \left( x \right) \exp \left( - \frac { 2 \pi inx } { T } \right) dx \nonumber \\
	& = & \int _0 ^\frac { T } { 2 } f \left( x \right) \exp \left( - \frac { 2 \pi inx } { T } \right) dx + \int _\frac { T } { 2 } ^T f \left( x \right) \exp \left( - \frac { 2 \pi inx } { T } \right) dx \nonumber \\
	& = & \int _0 ^\frac { T } { 2 } f \left( x \right) \exp \left( - \frac { 2 \pi inx } { T } \right) dx + \int _{ - \frac { T } { 2 } } ^0 f \left( x + T \right) \exp \left( - \frac { 2 \pi in \left( x + T \right) } { T } \right) dx \nonumber \\
	& = & \int _0 ^\frac { T } { 2 } f \left( x \right) \exp \left( - \frac { 2 \pi inx } { T } \right) dx + \int _{ - \frac { T } { 2 } } ^0 f \left( x + T \right) \exp \left( - \frac { 2 \pi inx} { T } + 2 \pi in \right) dx \nonumber \\
	& = & \int _0 ^\frac { T } { 2 } f \left( x \right) \exp \left( - \frac { 2 \pi inx } { T } \right) dx + \int _{ - \frac { T } { 2 } } ^0 f \left( x \right) \exp \left( - \frac { 2 \pi inx } { T } \right) dx \nonumber \\
	& = & \int _{ - \frac { T } { 2 } } ^\frac { T } { 2 } f \left( x \right) \exp \left( - \frac { 2 \pi inx } { T } \right) dx
\end{eqnarray}
ここで、$ \omega = \frac { 2 \pi n } { T } $とすると、
\begin{equation}
c_\frac { \omega T } { 2 \pi } = \int _a ^a f \left( x \right) \exp \left( - i \omega x \right) dx
\end{equation}
ここで、今までは波長$ T $の関数$ f \left( x \right) $を想定していたが、この$ T $を無限大に近づけると、
\begin{eqnarray}
	\lim _{ T \to \infty } c_\frac { \omega T } { 2 \pi } & = & \lim _{ T \to \infty } \int _0 ^T f \left( x \right) \exp \left( - i \omega x \right) dx \nonumber \\
	& = & \int _0 ^T f \left( x \right) \exp \left( - i \omega x \right) dx
\end{eqnarray}
\end{comment}
\begin{thebibliography}{9}
	\bibitem{FourierSeriesBibliography}高校数学の美しい物語 フーリエ級数展開の公式と意味$<$http://mathtrain.jp/fourierseries$>$
	\bibitem{ComplexFourierSeriesBibliography}高校数学の美しい物語 複素数型のフーリエ級数展開とその導出
\end{thebibliography}
\end{document}
