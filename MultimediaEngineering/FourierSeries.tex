\documentclass[a4paper]{jsarticle}
\usepackage[top=10truemm,bottom=10truemm,left=10truemm,right=10truemm]{geometry}
\usepackage{amsmath,amssymb}
\usepackage{type1cm}
\title{フーリエ級数展開}
\author{伊藤 太清}
\begin{document}
\maketitle
\section{フーリエ級数展開とは}
関数$f: \mathbb{R} \longrightarrow \mathbb{R}$がディリクレの条件を満たし、かつ周期$T$を持つ、すなわち$\forall x \in \mathbb{R} , f \left( x + T \right) = f \left( x \right)$を満たすとき、$f \left( x \right)$は、
\begin{equation}
\label{FourierSeries}
f \left( x \right) = \frac { a_0 } { 2 } + \sum _{ n = 1} ^{ \infty } \left( a_n \cos( \frac { 2 \pi nx } { T } ) + b_n \sin( \frac { 2 \pi nx } { T } ) \right)
\end{equation}
の形であらわすことが出来る。またこの形で表したとき、係数$a_n,b_n$は、
\begin{eqnarray}
a_n = \frac { 2 } { T } \int _0 ^T f \left( x \right) \cos( \frac { 2 \pi nx } { T } ) dx \\
b_n = \frac { 2 } { T } \int _0 ^T f \left( x \right) \sin( \frac { 2 \pi nx } { T } ) dx
\end{eqnarray}
となる。
\section{係数の導出}
フーリエ級数展開が成り立つ関数$f: \mathbb{R} \longrightarrow \mathbb{R}$について、
\begin{equation}
f \left( x \right) = \frac { a_0 } { 2 } + \sum _{ n = 1} ^{ \infty } \left( a_n \cos( \frac { 2 \pi nx } { T } ) + b_n \sin( \frac { 2 \pi nx } { T } ) \right)
\end{equation}
まず、$a_0$を導出する。\\
式\ref{FourierSeries}の両辺を$x \in \left[ 0,T \right]$で積分して、
\begin{eqnarray}
	\int _0 ^T f \left( x \right) dx &=& \int _0 ^T \left( \frac { a_0 } { 2 } + \sum _{ n = 1} ^{ \infty } \left( a_n \cos( \frac { 2 \pi nx } { T } ) + b_n \sin( \frac { 2 \pi nx } { T } ) \right) \right) dx \nonumber \\
	&=& \left[ \frac { a_0 } { 2 } + \sum _{ n = 1} ^\infty \left( a_n \frac { T } { 2 \pi n } \sin( \frac { 2 \pi nx } { T } ) - b_n \frac { T } { 2 \pi n } \cos( \frac { 2 \pi nx } { T } ) \right) \right] _{ x = 0 } ^{ x = T } \nonumber \\
	&=& \left( \frac { a_0T } { 2 } + \sum _{ n = 1 } ^\infty \left( \frac { a_0T } { 2 \pi n } \sin( 2 \pi n ) - \frac { b_nT } { 2 \pi n } \cos( 2 \pi n) \right) \right) - \left( \sum _{ m = 1 } ^\infty \left( - \frac { b_mT } { 2 \pi m } \right) \right) \nonumber \\
	&=& \frac { a_0T } { 2 } + \sum _{ n = 1 } ^\infty \left( - \frac { b_nT } { 2 \pi n } \right) - \sum _{ m = 1 } ^\infty \left( - \frac { b_mT } { 2 \pi m } \right) \nonumber \\
	&=& \frac { a_0T } { 2 }
\end{eqnarray}
よって、
\begin{equation}
a_0 = \frac { 2 } { T } \int _0 ^T f \left( x \right) dx
\end{equation}
次に、$a_m \left( m \in \mathbb{N} \right)$を導出する。
また、式\ref{FourierSeries}の両辺に$\cos(\frac { 2 \pi mx } { T })$をかけると、
\begin{equation}
	f \left( x \right) \cos(\frac { 2 \pi mx } { T }) = \sum _{ n = 1 } ^\infty \left( a_n \cos(\frac { 2 \pi nx } { T }) \cos(\frac { 2 \pi mx } { T }) + b_n \sin(\frac { 2 \pi nx } { T }) \cos(\frac { 2 \pi mx } { T }) \right)
\end{equation}
ここで、$\cos(\alpha x) \cos(\beta x)$と、$\sin(\alpha x) \cos(\beta x)$について考えてみる。\\
加法定理より、
\begin{eqnarray}
	\cos( \alpha x + \beta x ) = \cos( \alpha x ) \cos( \beta x ) - \sin( \alpha x ) \sin( \beta x ) \nonumber \\
	\cos( \alpha x - \beta x ) = \cos( \alpha x ) \cos( \beta x ) + \sin( \alpha x ) \sin( \beta x ) \nonumber
\end{eqnarray}
上の二つの式を足し合わせると、
\begin{equation}
	\cos( \alpha x + \beta x ) + \cos( \alpha x - \beta x ) = 2 \cos( \alpha x ) \cos( \beta x )
\end{equation}
よって、
\begin{equation}
	\frac { 1 } { 2 } \cos( \alpha x + \beta x ) + \frac { 1 } { 2 } \cos( \alpha x - \beta x ) = \cos( \alpha x ) \cos( \beta x )
\end{equation}
同様に、
\begin{eqnarray}
	\sin( \alpha x + \beta x ) = \sin( \alpha x ) \cos( \beta x ) + \cos( \alpha x ) \sin( \beta x ) \nonumber \\
	\sin( \alpha x - \beta x ) = \sin( \alpha x ) \cos( \beta x ) - \cos( \alpha x ) \sin( \beta x ) \nonumber
\end{eqnarray}
上の二つの式を足し合わせると、
\begin{equation}
	\sin( \alpha x + \beta x ) + \sin( \alpha x - \beta x ) = 2 \sin( \alpha x ) \cos( \beta x )
\end{equation}
よって、
\begin{equation}
	\frac { 1 } { 2 } \sin( \alpha x + \beta x ) + \frac { 1 } { 2 } \sin( \alpha x - \beta x ) = \sin( \alpha x ) \cos( \beta x )
\end{equation}
これらを適用すると、
\begin{eqnarray}
	&f& \left( x \right) \cos \frac { 2 \pi mx } { T } \nonumber \\
	&=& \frac { a_0 } { 2 } \cos \frac { 2 \pi mx } { T } \nonumber \\
	&+& \sum _{ n = 1} ^\infty \left( \frac { a_n } { 2 } \cos \frac { 2 \pi \left( n + m \right) x } { T } + \frac { a_n } { 2 } \cos \frac { 2 \pi \left( n - m \right) x } { T } + \frac { b_n } { 2 } \sin \frac { 2 \pi \left( n + m \right) x } { T } + \frac { b_n } { 2 } \sin \frac { 2 \pi \left( n - m \right) x } { T } \right)
\end{eqnarray}
両辺$x \in \left[ 0,T \right]$で積分すると、
\begin{eqnarray}
	&{\int _0 ^T}& f \left( x \right) \cos \frac { 2 \pi mx } { T } dx \nonumber \\
	&=& \frac { a_0 } { 2 } \int _0 ^T \cos \frac { 2 \pi mx } { T } dx \nonumber \\
	&+& \sum _{ n = 1 } ^\infty \left( \frac { a_n } { 2 } \int _0 ^T \cos \frac { 2 \pi \left( n + m \right) x } { T }  dx \right. \nonumber \\
	&+& \frac { a_n } { 2 } \int _0 ^T \cos \frac { 2 \pi \left( n - m \right) x } { T } dx \nonumber \\
	&+& \frac { b_n } { 2 } \int _0 ^T \sin \frac { 2 \pi \left( n + m \right) x } { T } dx \nonumber \\
	&+& \left. \frac { b_n } { 2 } \int _0 ^T \sin \frac { 2 \pi \left( n - m \right) x } { T } dx \right) \nonumber \\
	&=& \frac { a_0 } { 2 } \left[ \frac { T } { 2 \pi m } \sin \frac { 2 \pi mx } { T } \right] _{ x = 0 } ^{ x + T } \nonumber \\
	&+& \sum _{ n = 1 } ^\infty \left( \frac { a_n } { 2 } \left[ \frac { T } { 2 \pi \left( n + m \right) } \sin \frac { 2 \pi \left( n + m \right) x } { T } \right] _{ x = 0 } ^{ x = T } \right. \nonumber \\
	&+& \frac { a_n } { 2 } \left[ \frac { T } { 2 \pi \left( n - m \right) } \sin \frac { 2 \pi \left( n - m \right) x } { T } \right] _{ x = 0 } ^{ x = T } \nonumber \\
	&+& \frac { b_n } { 2 } \left[ \frac { T } { 2 \pi \left( n + m \right) } \cos \frac { 2 \pi \left( n + m \right) x } { T } \right] _{ x = 0 } ^{ x = T } \nonumber \\
	&+& \left. \frac { b_n } { 2 } \left[ \frac { T } { 2 \pi \left( n - m \right) } \cos \frac { 2 \pi \left( n - m \right) x } { T } \right] _{ x = 0 } ^{ x = T }\right) 
\end{eqnarray}
ここで、最初の項は$0$、シグマの中の四つの項についても、$n \neq m$のとき、いずれも$0$になるから、
\begin{eqnarray}
	&{\int _0 ^T}& f \left( x \right) \cos \frac { 2 \pi mx } { T } dx \nonumber \\
	&=& \int _0 ^T \left( \frac { a_0 } { 2 } \cos \frac { 2 \pi mx } { T } \right. \nonumber \\
	&+& \left. \sum _{ n = 1} ^\infty \left( \frac { a_n } { 2 } \cos \frac { 2 \pi \left( n + m \right) x } { T } + \frac { a_n } { 2 } \cos \frac { 2 \pi \left( n - m \right) x } { T } + \frac { b_n } { 2 } \sin \frac { 2 \pi \left( n + m \right) x } { T } + \frac { b_n } { 2 } \sin \frac { 2 \pi \left( n - m \right) x } { T } \right) \right) dx \nonumber \\
	&=& \int _0 ^T \left( a_m \cos ^2 \frac { 2 \pi mx } { T } + b_m \sin \frac { 2 \pi mx } { T } \cos \frac { 2 \pi mx } { T } \right) dx \nonumber \\
	&=& \int _0 ^T \left( a_m \left( \frac { 1 } { 2 } \left( 1 + \cos \frac { 4 \pi mx } { T } \right) \right) + \frac { b_m } { 2 } \sin \frac { 4 \pi mx } { T } \right) dx \nonumber \\
	&=& \int _0 ^T \left( \frac { a_m } { 2 } + \frac { a_m } { 2 } \cos \frac { 4 \pi mx } { T } + \frac { b_m } { 2 } \sin \frac { 4 \pi mx } { T } \right) dx \nonumber \\
	&=& \int _0 ^T \frac { a_m } { 2 } dx + \frac { a_m } { 2 } \int _0 ^T \cos \frac { 4 \pi mx } { T } dx + \frac { b_m } { 2 } \int _0 ^T \sin \frac { 4 \pi mx } { T } dx \nonumber \\
	&=& \left[ \frac { a_m } { 2 } x \right] _{ x = 0 } ^{ x = T } + \frac { a_m } { 2 } \left[ \frac { T } { 4 \pi m } \sin \frac { 4 \pi mx } { T } \right] _{ x = 0 } ^{ x = T } + \frac { b_m } { 2 } \left[ - \frac { T } { 4 \pi m } \cos \frac { 4 \pi mx } { T } \right] _{ x = 0 } ^{ x = T } \nonumber \\
	&=& \frac { a_mT } { 2 }
\end{eqnarray}
よって、
\begin{equation}
a_m = \frac { 2 } { T } \int _0 ^T f \left( x \right) \cos \frac { 2 \pi mx } { T } dx
\end{equation}
同様に、式\ref{FourierSeries}の両辺に$\sin \frac { 2 \pi mx } { T }$をかけ、$x \in \left[ 0,T \right]$で積分することによって、
\begin{equation}
b_m = \frac { 2 } { T } \int _0 ^T f \left( x \right) \sin \frac { 2 \pi mx } { T } dx
\end{equation}
を得ることが出来る。
\end{document}
